\noindent Information provided below refers to hardware and software used during the evaluation of performances.

\subsection{Hardware}
Several machines were used to make the measurements. In the following, the following coding will be used:
\begin{itemize}
  \item \textbf{Machine A}: Dell XPS17 laptop PC
  \item \textbf{Machine B}: HP Elitebook laptop
  \item \textbf{Machine C} A desktop PC
  \item \textbf{Machine P4}: A raspberry Pi4
  \item \textbf{Machine P3}: A raspberry Pi3
\end{itemize}
In particular:
\begin{itemize}
    \item for the measurements of the \verb|sequential|, \verb|sequential_pre|, \verb|cuda| and \verb|cuda_texture| programs the machine A was used;
    \item for the measurements of the \verb|mpi| and \verb|cuda_mpi| programs, a cluster consisting of the machine A (master), the machines B and C (slaves) was used.
\end{itemize}
On the next pages we report the CPU, memory specifications of each machine as well as the GPU specifications of machine A.

\subsubsection{Machine A}
\paragraph{CPU}\mbox{}\\
This CPU has 20 processors, but in order not to be redundant we only present the specifications relative to a single core.
{\fontsize{9}{10}\selectfont \verbatiminput{infopcmichele.txt}}
\paragraph{Memory}
{\fontsize{9}{10}\selectfont \verbatiminput{infopcmemmichele.txt}}
\paragraph{GPU}
{\fontsize{9}{10}\selectfont \verbatiminput{gpumicheleinfo1.txt}}
{\fontsize{9}{10}\selectfont \verbatiminput{gpumicheleinfo2.txt}}

\subsubsection{Machine B}
\paragraph{CPU}\mbox{}\\
This CPU has 16 processors, but in order not to be redundant we only present the specifications relative to a single core.
{\fontsize{9}{10}\selectfont \verbatiminput{antonio-hp-cpuinfo.txt}}
\paragraph{Memory}
{\fontsize{9}{10}\selectfont \verbatiminput{antonio-hp-meminfo.txt}}

\subsubsection{Machine C}
\paragraph{CPU}\mbox{}\\
This CPU has 8 processors, but in order not to be redundant we only present the specifications relative to a single core.
{\fontsize{9}{10}\selectfont \verbatiminput{cartagho_cpu_info.txt}}
\paragraph{Memory}
{\fontsize{9}{10}\selectfont \verbatiminput{cartagho_mem_info.txt}}

\subsubsection{Machine P4}
\paragraph{CPU}\mbox{}\\
This CPU has 4 processors, but in order not to be redundant we only present the specifications relative to a single core.
{\fontsize{9}{10}\selectfont \verbatiminput{raspPI4.txt}}
\paragraph{Memory}
{\fontsize{9}{10}\selectfont \verbatiminput{raspPI4mem.txt}}

\subsubsection{Machine P3}
\paragraph{CPU}\mbox{}\\
This CPU has 4 processors, but in order not to be redundant we only present the specifications relative to a single core.
{\fontsize{9}{10}\selectfont \verbatiminput{raspPI3.txt}}
\paragraph{Memory}
{\fontsize{9}{10}\selectfont \verbatiminput{raspPI3mem.txt}}

\subsection{Software}
\noindent Every machine used runs \textbf{Ubuntu 22.04 LTS} as its operating system. \\The used compilers were:
\begin{itemize}
    \item \textbf{C compiler}: gcc 11.3.0;
    \item \textbf{MPI framework}: MPICH 4.0;
    \item \textbf{CUDA compiler}: Cuda compilation tools, release 12.0, V12.0.76, Build cuda\_12.0.r12.0/compiler.31968024\_0.
\end{itemize}
In addition, Python 3.10.6 was used for creating the tables and speedup graphs based on the measurements.