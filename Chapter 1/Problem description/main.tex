\label{alg:tarjan_theorical}
The goal of this project is to parallelize and evaluate performances of Tarjan's Algorithm, mainly by using MPI and CUDA. 
\textbf{Tarjan's strongly connected components algorithm} is an algorithm in graph theory for finding the strongly connected components (SCCs) of a directed graph.
It runs in \textbf{O(n+m)}, where n is the number of vertices and m is the number of edges, thanks to a depth-first traversal procedure to output the list of all SCCs for a given directed graph \cite{doi:10.1137/0201010}.

In the mathematical theory of directed graphs, a graph is said to be \textbf{strongly connected} if every vertex is reachable from every other vertex.
The \textbf{strongly connected components} of an arbitrary directed graph form a partition into subgraphs that are themselves strongly connected \cite{wiki:SCC}.

The algorithm takes a directed graph as input, and produces a partition of the graph's vertices into the graph's strongly connected components. 
Each vertex of the graph appears in exactly one of the strongly connected components. Any vertex that is not on a directed cycle forms a strongly connected component all by itself: for example, a vertex whose in-degree or out-degree is 0, or any vertex of an acyclic graph.

The basic idea of the algorithm is this: a depth-first search (DFS) begins from an arbitrary start node (and subsequent depth-first searches are conducted on any nodes that have not yet been found).
As usual with DFS, the search visits every node of the graph exactly once, declining to revisit any node that has already been visited. Thus, the collection of search trees is a spanning forest of the graph.
The strongly connected components will be recovered as certain subtrees of this forest. The roots of these subtrees are called the "roots" of the strongly connected components.
Any node of a strongly connected component might serve as a root, if it happens to be the first node of a component that is discovered by search \cite{wiki:Tarjan}.

The nodes are thus placed in a stack in the order in which they are visited. When the search moves up from a visited subtree, the corresponding nodes are taken from the stack, and it is determined whether each node is the root of a strongly connected component.
If it is, then it forms a strongly connected component together with all those taken before.

