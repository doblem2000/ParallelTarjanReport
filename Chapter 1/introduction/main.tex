The assigned problem is to provide a parallel version of the Tarjan's algorithm to find Strongly Connected Components (SCC) in a graph with an implementation that had to use an hybrid message passing/CUDA paradigm, implemented by using MPI and CUDA.
In particular, in this report different implementations of the Tarjan's algorithm will be presented, and their performances will be evaluated:
\begin{itemize}
    \item a sequential version described in \ref{alg:sequential};
    \item a sequential version with graph preprocessing described in \ref{alg:sequential_pre};
    \item a parallel version with MPI described in \ref{alg:mpi_only};
    \item a parallel version with CUDA described in \ref{alg:cuda_only};
    \item a parallel version based on a hybrid CUDA/MPI approach described in \ref{alg:cuda_mpi}.
    \item a parallel version with CUDA which exploits texture memory described in \ref{alg:cuda_texture};
\end{itemize}

In the following pages the problem faced is going to be described in a detailed way, along with the most important theoretical concepts about GPUs (Graphics Processing Unit), 
CUDA and MPI.
Great attention is going to be reserved to the description of the different case studies considered. 
%the results are going to be analysed and explained in the light of the theoretical knowledge acquired during the High-Performance Computing course held by prof. Francesco Moscato at University of Salerno.
