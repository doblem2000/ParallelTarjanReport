\label{alg:sequential}
We decided to represent graphs with the adjacency map representation for performance reasons. This representation is based on the adjacency list representation \cite{wiki:Adjacency_list}, but it uses hash tables \cite{attractivechaos_2018} instead of lists. Our sequential implementation of Tarjan's algorithm is based on the algorithm described in \cite{geeksforgeeks}. The main difference is that we implemented \verb|disc|, \verb|low| and \verb|stackMember| data structures as hash tables instead of arrays. This is necessary because we represented graphs with the adjacency map representation. This fact gives us an important advantage: we don't have to put constraints on the IDs of the node of the graph. In other words, the implementation from geeksforgeeks requires that vertices are numbered from $0$ to $N-1$ while we can run our implementation on a graph with arbitrary node IDs. This is a great advantage because in our MPI implementation (described in \ref{alg:mpi_only}), slave processes will execute this algorithm on arbitrary subgraphs.